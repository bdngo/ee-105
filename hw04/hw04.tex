\documentclass{article}
\usepackage{../cs170}
\AtBeginDocument{\RenewCommandCopy\qty\SI}

\def\title{HW 04}

\begin{document}

\maketitle

\question{}

\begin{center}
    \begin{circuitikz}
        \draw (0, 0) node[ground]{} to[V, l=\(V_s\), invert] (0, 2) to[R, l2=\(R_1\) and \qty{100}{\kilo\ohm}, l2 valign=b] (2, 2) to[C, l2_=\(C_1\) and \qty{10}{\pico\farad}, v^=\(V_i\)] (2, 0) node[ground]{};
        \draw (4, 0) node[ground]{} to[cI, l_=\(G_m V_i\), invert] (4, 2) to[short] (6, 2) to[C, l2=\(C_2\) and \qty{100}{\nano\farad}, l2 valign=b] (8, 2) to[short] (10, 2) node[ocirc]{};
        \draw (6, 0) node[ground]{} to[R, l2_=\(R_2\) and \qty{100}{\kilo\ohm}] (6, 2);
        \draw (8, 0) node[ground]{} to[R, l2_=\(R_3\) and \qty{10}{\kilo\ohm}] (8, 2);
        \draw (10, 2) to[open, v=\(V_o\)] (10, 0);
    \end{circuitikz}
\end{center}
By inspection, we can see that \(T_i(s)\) is a simple RC low-pass filter, so we have
\begin{equation}
    T_i(s) = \frac{1}{1 + s R_1 C_1}
\end{equation}
Performing node voltage analysis on the second circuit in the \(s\)-domain (assuming the current through \(C_2\) is going left to right), we have
\begin{align}
    \label{eq:q1} G_m V_i + \frac{u_1}{R_2} + \frac{u_1 - V_o}{\frac{1}{s C_2}} &= 0 \\
    \frac{u_1 - V_o}{\frac{1}{s C_2}} &= \frac{V_o}{R_3} \\
    \implies u_1 &= V_o \left(1 + \frac{1}{s R_3 C_2}\right)
\end{align}
Substituting this expression back into \autoref{eq:q1},
\begin{align}
    G_m V_i + \frac{V_o}{R_2} \left(1 + \frac{1}{s R_3 C_2}\right) + s C_2 \left(\cancel{V_o} + \frac{V_o}{s R_3 C_2} - \cancel{V_o}\right) &= 0 \\
    G_m V_i + V_o \left(\frac{1}{R_2} + \frac{1}{R_3} + \frac{1}{s R_2 R_3 C_2}\right) &= 0 \\
    \implies T_o(s) = \frac{V_o}{V_i} &= -\frac{G_m}{\frac{1}{R_2} + \frac{1}{R_3} + \frac{1}{s R_2 R_3 C_2}} \\
    \implies T(s) = T_i(s) T_o(s) &= -\frac{G_m}{(1 + s R_1 C_1) \left(\frac{1}{R_2} + \frac{1}{R_3} + \frac{1}{s R_2 R_3 C_2}\right)}
\end{align}

\question{}

\begin{center}
    \includegraphics[width=0.7\textwidth]{Screenshot_2023-02-11_18-20-53.png}
\end{center}
By inspection, we have poles at \(\omega = 0.1, 2, 20 \, \unit{\radian\per\second}\).
We also have zeroes at \(\omega = \qty{0.5}{\radian\per\second}\).
We also have a DC gain of \qty{20}{\deci\bel}.
The transfer function is
\begin{equation}
    H(s) = 80 \frac{s + 0.5}{(s + 0.1) (s + 2) (s + 20)}
\end{equation}

\question{}

\begin{center}
    \begin{circuitikz}\draw
        (0, 0) node[op amp, noinv input up](oa){}
        (oa.+) to[short] ++(-2, 0) to[C, l_=\(C_1\)] ++(-2, 0) to[sV, l_=\(V_s(t)\)] ++(0, -4) to[short] ++(2, 0) node[ground](gnd){} to[R, l_=\(R_1\)] ++(0, 4)
        (oa.-) to[short] ++(0, -1) node[](fb){} to[short] (fb -| oa.out) to[short] (oa.out)
        (oa.out) to[C, l=\(C_2\)] ++(2, 0) node[](vout){} to[R, l_=\(R_2\), v^=\(V_o(t)\)] (gnd -| vout) to[short] (gnd)
    ;\end{circuitikz}
\end{center}
In the \(s\)-domain,
\begin{center}
    \begin{circuitikz}\draw
        (0, 0) node[op amp, noinv input up](oa){}
        (oa.+) to[short] ++(-2, 0) to[generic, l_=\(\frac{1}{s C_1}\), v<=\(\)] ++(-2, 0) to[sV, l_=\(V_s(s)\), v] ++(0, -4) to[short] ++(2, 0) node[ground](gnd){} to[generic, l_=\(R_1\), v<=\(\)] ++(0, 4)
        (oa.-) to[short] ++(0, -1) node[](fb){} to[short] (fb -| oa.out) to[short] (oa.out)
        (oa.out) to[generic, l=\(\frac{1}{s C_2}\), v=\(\)] ++(2, 0) node[](vout){} to[generic, l_=\(R_2\), v^=\(V_o(s)\)] (gnd -| vout) to[short] (gnd)
    ;\end{circuitikz}
\end{center}
Using the voltage divider formula, we have
\begin{align}
    V^+ &= \frac{s R_1 C_1}{1 + s R_1 C_1} V_s(s) \\
    V_o(s) &= \frac{s R_2 C_2}{1 + s R_2 C_2} V^-
\end{align}
By the golden op-amp rules, \(V^+ = V^-\), so
\begin{equation}
    H(s) = \frac{V_o(s)}{V_s(s)} = \frac{s R_2 C_2}{1 + s R_2 C_2} \frac{s R_1 C_1}{1 + s R_1 C_1}
\end{equation}

\question{}

\begin{align}
    x(t) &=
    \begin{cases}
        t + 1 & 0 \leqslant t \leqslant 1 \\
        2 - t & 1 < t \leqslant 2 \\
        0 & \text{elsewhere}
    \end{cases} \\
    h(t) &= \delta(t + 2) + 2 \delta(t + 1)
\end{align}
The convolution is
\begin{equation}
    x(t) \ast h(t) = \int_\R x(\tau) h(t - \tau) \, \dd{\tau} = \int_\R x(\tau) \delta(t + 2 - \tau) \, \dd{\tau} + 2 \int_\R x(\tau) \delta(t + 1 - \tau) \, \dd{\tau}
\end{equation}
By the sifting property, we have
\begin{align}
    x(t) \ast h(t) = x(t + 2) + 2x(t + 1) &=
    \begin{cases}
        t + 3 & -2 \leqslant t \leqslant -1 \\
        -t & -1 < t \leqslant 0 \\
        0 & \text{elsewhere}
    \end{cases} +
    \begin{cases}
        2t + 4 & -1 \leqslant t \leqslant 0 \\
        2 - 2t & 0 < t \leqslant 1 \\
        0 & \text{elsewhere}
    \end{cases} \\
    &=
    \begin{cases}
        t + 3 & -2 \leqslant t < -1 \\
        4 & t = -1 \\
        t + 4 & -1 \leqslant t \leqslant 0 \\
        2 - 2t & 0 < t \leqslant 1 \\
        0 & \text{elsewhere}
    \end{cases}
\end{align}
\begin{center}
    \resizebox{0.8\textwidth}{!}{\input{q4.pgf}}
\end{center}

\question{}

\begin{align}
    x(t) &= u(t - 3) - u(t - 5) \\
    h(t) &= e^{-3t} u(t)
\end{align}

\begin{subparts}
    \item \begin{align}
        y(t) &= \int_\R (u(\tau - 3) - u(\tau - 5)) e^{-3 (t - \tau)} u(t - \tau) \, \dd{\tau} \\
        &= e^{-3t} \int_\R e^{3 \tau} u(\tau - 3) u(t - \tau) \, \dd{\tau} - e^{-3t} \int_\R e^{3 \tau} u(\tau - 5) u(t - \tau) \, \dd{\tau} \\
        &= e^{-3t} \left(\int_3^t e^{3 \tau} \, \dd{\tau} - \int_5^t e^{3 \tau} \, \dd{\tau}\right) \\
        &= \frac{1}{3} e^{-3t} \left((e^{3t} - e^9) u(t - 3) - (e^{3t} - e^{15}) u(t - 5)\right) \\
        &= \frac{1}{3} \left(\left(1 - e^{-3 (t - 3)}\right) u(t - 3) - \left(1 - e^{-3 (t - 5)}\right) u(t - 5)\right)
    \end{align}
    \item \begin{align}
        y(t) &= \int_\R (\delta(\tau - 3) - \delta(\tau - 5)) e^{-3 (t - \tau)} u(t - \tau) \, \dd{\tau} \\
        &= e^{-3t} \int_\R e^{3 \tau} \delta(\tau - 3) u(t - \tau) \, \dd{\tau} - e^{-3t} \int_\R e^{3 \tau} \delta(\tau - 5) u(t - \tau) \, \dd{\tau} \\
        &= e^{-3t} (e^9 u(t - 3) - e^{15} u(t - 5)) \\
        &= e^{-3 (t - 3)} u(t - 3) - e^{-3 (t - 5)} u(t - 5)
    \end{align}
    \item We have that \(y(t) = \int_3^t g(\tau) \, \dd{\tau}\) for \(3 \leqslant t \leqslant 5\).
    \item
    \begin{enumerate}
        \item In the \(s\)-domain,
        \begin{align}
            Y(s) = X(s) H(s) &= \left(\frac{e^{-3s}}{s} - \frac{e^{-5s}}{s}\right) \left(\frac{1}{s + 3}\right) \\
            &= \frac{e^{-3s}}{s (s + 3)} - \frac{e^{-5s}}{s (s + 3)} \\
            \overset{\mathcal{L}^{-1}}{\implies} y(t) &= \frac{1}{3} (1 - e^{-3 (t - 3)}) u(t - 3) - \frac{1}{3} (1 - e^{-3 (t - 5)}) u(t - 5)
        \end{align}
        \item In the \(s\)-domain,
        \begin{align}
            G(s) &= (s X(s) - \cancelto{0}{x(0^-)}) H(s) = s X(s) H(s) \\
            &= \frac{e^{-3s}}{s + 3} - \frac{e^{-5s}}{s + 3} \\
            \overset{\mathcal{L}^{-1}}{\implies} g(t) &= e^{-3t} u(t - 3) - e^{-3t} u(t - 5)
        \end{align}
    \end{enumerate}
\end{subparts}

\end{document}
