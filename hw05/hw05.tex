\documentclass{article}
\usepackage{../cs170}
\AtBeginDocument{\RenewCommandCopy\qty\SI}

\def\title{HW 05}

\begin{document}

\maketitle

\question{}

\begin{center}
    \begin{circuitikz}\draw
        (0, 0) node[op amp](oa1){}
        (oa1.+) to[V, l=\(V_{\text{bias}}\)] ++(0, -2) node[ground]{}
        (oa1.-) to[C, l_=\(C_1\)] ++(-2, 0) to[R, l_=\(R_s\)] ++(-2, 0) to[sV, v=\(V_i\)] ++(0, -2) node[ground]{}
        (oa1.-) to[open] ++(0, 1) node[](c1){} to[C, l=\(C_2\)] (c1 -| oa1.out)
        (oa1.-) to[short] ++(0, 3) node[](r1){} to[R, l=\(R_1\)] (r1 -| oa1.out) to[short] (oa1.out)
        (oa1.out)++(2, -0.5) node[op amp, noinv input up](oa2){}
        (oa1.out) to[short] (oa2.+)
        (oa2.-) to[short] ++(0, -1) coordinate(fb) to[R, l=\(R_2\)] ++(0, -2) node[ground]{}
        (fb) to[C, l=\(C_3\)] (fb -| oa2.out) to[short] (oa2.out) to[short] ++(1, 0) node[ocirc, label=right:\(V_o\)]{}
    ;\end{circuitikz}
\end{center}
In the \(s\)-domain,
\begin{center}
    \begin{circuitikz}\draw
        (0, 0) node[op amp](oa1){}
        (oa1.+) to[V, l=\(V_{\text{bias}} \delta(s)\)] ++(0, -2) node[ground]{}
        (oa1.-) to[generic, l_=\(R_s + \frac{1}{s C_1}\)] ++(-2, 0) to[sV, v=\(V_i(s)\)] ++(0, -2) node[ground]{}
        (oa1.-) to[short] ++(0, 1) node[](c1){} to[generic, l=\(\frac{R_1}{1 + s R_1 C_2}\)] (c1 -| oa1.out) to[short] (oa1.out)
        (oa1.out)++(2, -0.5) node[op amp, noinv input up](oa2){}
        (oa1.out) to[short] (oa2.+)
        (oa2.-) to[short] ++(0, -1) coordinate(fb) to[generic, l=\(R_2\)] ++(0, -2) node[ground]{}
        (fb) to[generic, l=\(\frac{1}{s C_3}\)] (fb -| oa2.out) to[short] (oa2.out) to[short] ++(1, 0) node[ocirc, label=right:\(V_o(s)\)]{}
    ;\end{circuitikz}
\end{center}
Using superposition, we can derive the output of the first op-amp as
\begin{align}
    V_x(s) &= -V_i(s) \frac{\frac{R_1}{1 + s R_1 C_2}}{\frac{1 + s R_s C_1}{s C_1}} + V_{\text{bias}} \delta(s) \left(1 + \frac{\frac{R_1}{1 + s R_1 C_2}}{\frac{1 + s R_s C_1}{s C_1}}\right) \\
    &= -V_i(s) \frac{s R_1 C_1}{(1 + s R_1 C_2) (1 + s R_s C_1)} + V_{\text{bias}} \delta(s) \left(1 + \frac{s R_1 C_1}{(1 + s R_1 C_2) (1 + s R_s C_1)}\right)
\end{align}
We can then multiply by the second op-amp's transfer function to get
\begin{align}
    V_o(s) &= V_x \left(1 + \frac{1}{s R_2 C_3}\right) \\
           &= \left(-V_i(s) \frac{s R_1 C_1}{(1 + s R_1 C_2) (1 + s R_s C_1)} + \cancelto{0}{V_{\text{bias}}} \delta(s) \left(1 + \frac{s R_1 C_1}{(1 + s R_1 C_2) (1 + s R_s C_1)}\right)\right) \left(1 + \frac{1}{s R_2 C_3}\right) \\
    \implies \frac{V_o(s)}{V_i(s)} &= -\frac{s R_1 C_1}{(1 + s R_1 C_2) (1 + s R_s C_1)} \left(1 + \frac{1}{s R_2 C_3}\right)
\end{align}

\question{}

\begin{center}
    \begin{circuitikz}\draw
        (0, 0) node[op amp](oa){}
        (oa.-) to[R, l_=\(R_s\)] ++(-2, 0) to[sV, l_=\(V_i\)] ++(0, -2) node[ground]{}
        (oa.+) to[short] ++(0, -1) node[ground]{}
        (oa.-) to[short] ++(0, 1) coordinate(c1) to[C, l=\(C_1\)] (c1 -| oa.out) to[short] (oa.out) to[short] ++(1, 0) node[ocirc, label=right:\(V_o\)]{}
    ;\end{circuitikz}
\end{center}

\begin{subparts}
    \item Applying a test voltage at the input, the current through \(R_s\) is \(I_{test} = \frac{V_{test} - \cancelto{0}{V^-}}{R_s}\), where we use the ideality of the op amp to cancel out \(V^-\).
    This means that \(Z_{test} = \frac{V_{test}}{I_{test}} = R_s\).
    Applying a test voltage at the output, since we are applying a DC source, the voltage difference at the capacitor is \(0\) due to the virtual short, therefore we have \(Z_{out} = 0\).
    \item The circuit in the \(s\)-domain is now
    \begin{center}
        \begin{circuitikz}\draw
            (0, 0) node[ground]{} to[sV, l=\(V_{test}\), i^>=\(I_{test}\)] ++(0, 2) to[generic, l=\(R_s\)] ++(2, 0) coordinate(u1) to[open, v<=\(V_{in}\)] ++(0, -2) node[ground]{}
            (5, 0) node[ground]{} to[cV, l=\(A V_{in}\), invert] ++(0, 2) to[generic, l=\(R_{out}\)] ++(2, 0) coordinate(u2) to[short] ++(1, 0) node[ocirc, label=right:\(V_o\)]{}
            (u1) to[short] ++(0, 1) coordinate(u3) to[generic, l=\(\frac{1}{s C_1}\)] (u3 -| u2) to[short] (u2)
        ;\end{circuitikz}
    \end{center}
    We can create the loop equation
    \begin{align}
        V_{test} - I_{test} R_s - I_{test} \frac{1}{s C_1} - I_{test} R_{out} + A (V_{test} - I_{test} R_s) &= 0 \\
        V_{test} (1 + A) - I_{test} \left(R_s + \frac{1}{s C_1} - R_{out} - A R_s\right) &= 0 \\
        Z_{in} = \frac{V_{test}}{I_{test}} &= \frac{R_s + \frac{1}{s C_1} - R_{out} - A R_s}{1 + A}
    \end{align}
    For the output impedance, we can apply a test voltage at \(V_o\) and null \(V_i\) to get
    \begin{center}
        \begin{circuitikz}\draw
            (0, 0) node[ground]{} to[short] ++(0, 2) to[generic, l=\(R_s\)] ++(2, 0) coordinate(u1) to[open, v<=\(V_{in}\)] ++(0, -2) node[ground]{}
            (5, 0) node[ground]{} to[cV, l=\(A V_{in}\), invert] ++(0, 2) to[generic, l=\(R_{out}\), i^<=\(I_2\)] ++(2, 0) coordinate(u2) to[short] ++(1, 0) node[circ, label=right:\(V_o\)]{} coordinate(vo)
            (u1) to[short] ++(0, 1) coordinate(u3) to[generic, l=\(\frac{1}{s C_1}\)] (u3 -| u2) to[short, i<=\(I_1\)] (u2)
            (vo) to[sV, l=\(V_{test}\), i=\(I_{test}\)] ++(0, -2) node[ground]{}
        ;\end{circuitikz}
    \end{center}
    We can obtain the node equations
    \begin{align}
        I_{test} &= \frac{V_{test} - A V_{in}}{R_{out}} + s C_1 (V_{test} + V_{in}) \\
        V_{in} &= V_{test} - s R_s C_1 (V_{test} + V_{in}) \\
        \implies V_{in} &= V_{test} \frac{1 - s R_s C_1}{1 + s R_s C_1} \\
        I_{test} &= V_{test} \left(\frac{1}{R_{out}} - \frac{A}{R_{out}} \frac{1 - s R_s C_1}{1 + s R_s C_1} + sC_1 \left(1 + \frac{1 - s R_s C_1}{1 + s R_s C_1}\right)\right) \\
        \implies \frac{1}{Z_{out}} &= \left(\frac{1}{R_{out}} - \frac{A}{R_{out}} \frac{1 - s R_s C_1}{1 + s R_s C_1} + sC_1 \left(1 + \frac{1 - s R_s C_1}{1 + s R_s C_1}\right)\right)
    \end{align}
    When \(A \to \infty\), we have \(G_{out} = \infty \implies Z_{out} = 0\) and \(Z_{in} = R_s\).
    \item The answer can be obtained similarly by substituting the constant gain \(A\) with the gain \(\frac{A}{1 + j \frac{\omega}{\omega_0}}\).
\end{subparts}

\question{}

\begin{center}
    \begin{circuitikz}\draw
        (0, 0) node[op amp](oa){}
        (oa.+) to[short] ++(-1, 0) to[R, l=\qty{1}{\kilo\ohm}] ++(-2, 0) to[sV, l_=\(V_i\)] ++(0, -2) node[ground]{}
        (oa.+)++(-1, 0) to[I, l=\qty{0.1}{\milli\ampere}, invert] ++(0, -2) node[ground]{}
        (oa.-) to[short] ++(-1, 0) to[V, l_=\qty{0.1}{\volt}, invert] ++(-2, 0) to[short] ++(0, 3) coordinate(fb) to[R, l=\qty{2}{\kilo\ohm}] (fb -| oa.out) to[short] (oa.out)
        (fb) to[R, l=\qty{1}{\kilo\ohm}] ++(-2, 0) node[ground]{}
        (oa.-)++(-1, 0) to[I, l_=\qty{0.2}{\milli\ampere}, invert] ++(0, 2) node[ground, rotate=180]{}
        (oa.out) to[short] ++(1, 0) node[ocirc, label=right:\(V_o\)]{}
    ;\end{circuitikz}
\end{center}
Slew rate: \qty{2}{\volt\per\micro\second} \\
Signal:
\begin{equation}
    x(t) =
    \begin{cases}
        \qty{1}{\volt} & 0 \leqslant t \leqslant \qty{4}{\micro\second} \\
        0 & \text{elsewhere}
    \end{cases}
\end{equation}
We can define the node equations
\begin{align}
    \qty{0.2}{\milli\ampere} &= \frac{V^- + \qty{0.1}{\volt} - V_o}{\qty{2}{\kilo\ohm}} + \frac{V^- + \qty{0.1}{\kilo\ohm}}{\qty{1}{\kilo\ohm}} \\
    V_i + (\qty{0.1}{\milli\ampere}) (\qty{1}{\kilo\ohm}) &= V^+ \\
    V^+ &= V^- \\
    \implies \qty{0.2}{\milli\ampere} &= \frac{V_i + \qty{0.1}{\volt} + \qty{0.1}{\volt} - V_o}{\qty{2}{\kilo\ohm}} + \frac{V_i + \qty{0.1}{\volt} + \qty{0.1}{\volt}}{\qty{1}{\kilo\ohm}} \\
    \qty{0.4}{\volt} &= V_i + \qty{0.2}{\volt} - V_o + 2 V_i + \qty{0.4}{\volt} \\
  \implies V_o &= 3 V_i + \qty{0.2}{\volt}
\end{align}
\begin{center}
    \resizebox{0.8\textwidth}{!}{%% Creator: Matplotlib, PGF backend
%%
%% To include the figure in your LaTeX document, write
%%   \input{<filename>.pgf}
%%
%% Make sure the required packages are loaded in your preamble
%%   \usepackage{pgf}
%%
%% Also ensure that all the required font packages are loaded; for instance,
%% the lmodern package is sometimes necessary when using math font.
%%   \usepackage{lmodern}
%%
%% Figures using additional raster images can only be included by \input if
%% they are in the same directory as the main LaTeX file. For loading figures
%% from other directories you can use the `import` package
%%   \usepackage{import}
%%
%% and then include the figures with
%%   \import{<path to file>}{<filename>.pgf}
%%
%% Matplotlib used the following preamble
%%   \usepackage{fontspec}
%%
\begingroup%
\makeatletter%
\begin{pgfpicture}%
\pgfpathrectangle{\pgfpointorigin}{\pgfqpoint{6.400000in}{4.800000in}}%
\pgfusepath{use as bounding box, clip}%
\begin{pgfscope}%
\pgfsetbuttcap%
\pgfsetmiterjoin%
\definecolor{currentfill}{rgb}{1.000000,1.000000,1.000000}%
\pgfsetfillcolor{currentfill}%
\pgfsetlinewidth{0.000000pt}%
\definecolor{currentstroke}{rgb}{1.000000,1.000000,1.000000}%
\pgfsetstrokecolor{currentstroke}%
\pgfsetdash{}{0pt}%
\pgfpathmoveto{\pgfqpoint{0.000000in}{0.000000in}}%
\pgfpathlineto{\pgfqpoint{6.400000in}{0.000000in}}%
\pgfpathlineto{\pgfqpoint{6.400000in}{4.800000in}}%
\pgfpathlineto{\pgfqpoint{0.000000in}{4.800000in}}%
\pgfpathlineto{\pgfqpoint{0.000000in}{0.000000in}}%
\pgfpathclose%
\pgfusepath{fill}%
\end{pgfscope}%
\begin{pgfscope}%
\pgfsetbuttcap%
\pgfsetmiterjoin%
\definecolor{currentfill}{rgb}{1.000000,1.000000,1.000000}%
\pgfsetfillcolor{currentfill}%
\pgfsetlinewidth{0.000000pt}%
\definecolor{currentstroke}{rgb}{0.000000,0.000000,0.000000}%
\pgfsetstrokecolor{currentstroke}%
\pgfsetstrokeopacity{0.000000}%
\pgfsetdash{}{0pt}%
\pgfpathmoveto{\pgfqpoint{0.800000in}{0.528000in}}%
\pgfpathlineto{\pgfqpoint{5.760000in}{0.528000in}}%
\pgfpathlineto{\pgfqpoint{5.760000in}{4.224000in}}%
\pgfpathlineto{\pgfqpoint{0.800000in}{4.224000in}}%
\pgfpathlineto{\pgfqpoint{0.800000in}{0.528000in}}%
\pgfpathclose%
\pgfusepath{fill}%
\end{pgfscope}%
\begin{pgfscope}%
\pgfsetbuttcap%
\pgfsetroundjoin%
\definecolor{currentfill}{rgb}{0.000000,0.000000,0.000000}%
\pgfsetfillcolor{currentfill}%
\pgfsetlinewidth{0.803000pt}%
\definecolor{currentstroke}{rgb}{0.000000,0.000000,0.000000}%
\pgfsetstrokecolor{currentstroke}%
\pgfsetdash{}{0pt}%
\pgfsys@defobject{currentmarker}{\pgfqpoint{0.000000in}{-0.048611in}}{\pgfqpoint{0.000000in}{0.000000in}}{%
\pgfpathmoveto{\pgfqpoint{0.000000in}{0.000000in}}%
\pgfpathlineto{\pgfqpoint{0.000000in}{-0.048611in}}%
\pgfusepath{stroke,fill}%
}%
\begin{pgfscope}%
\pgfsys@transformshift{1.025455in}{0.528000in}%
\pgfsys@useobject{currentmarker}{}%
\end{pgfscope}%
\end{pgfscope}%
\begin{pgfscope}%
\definecolor{textcolor}{rgb}{0.000000,0.000000,0.000000}%
\pgfsetstrokecolor{textcolor}%
\pgfsetfillcolor{textcolor}%
\pgftext[x=1.025455in,y=0.430778in,,top]{\color{textcolor}\rmfamily\fontsize{10.000000}{12.000000}\selectfont \(\displaystyle {0}\)}%
\end{pgfscope}%
\begin{pgfscope}%
\pgfsetbuttcap%
\pgfsetroundjoin%
\definecolor{currentfill}{rgb}{0.000000,0.000000,0.000000}%
\pgfsetfillcolor{currentfill}%
\pgfsetlinewidth{0.803000pt}%
\definecolor{currentstroke}{rgb}{0.000000,0.000000,0.000000}%
\pgfsetstrokecolor{currentstroke}%
\pgfsetdash{}{0pt}%
\pgfsys@defobject{currentmarker}{\pgfqpoint{0.000000in}{-0.048611in}}{\pgfqpoint{0.000000in}{0.000000in}}{%
\pgfpathmoveto{\pgfqpoint{0.000000in}{0.000000in}}%
\pgfpathlineto{\pgfqpoint{0.000000in}{-0.048611in}}%
\pgfusepath{stroke,fill}%
}%
\begin{pgfscope}%
\pgfsys@transformshift{1.927273in}{0.528000in}%
\pgfsys@useobject{currentmarker}{}%
\end{pgfscope}%
\end{pgfscope}%
\begin{pgfscope}%
\definecolor{textcolor}{rgb}{0.000000,0.000000,0.000000}%
\pgfsetstrokecolor{textcolor}%
\pgfsetfillcolor{textcolor}%
\pgftext[x=1.927273in,y=0.430778in,,top]{\color{textcolor}\rmfamily\fontsize{10.000000}{12.000000}\selectfont \(\displaystyle {2}\)}%
\end{pgfscope}%
\begin{pgfscope}%
\pgfsetbuttcap%
\pgfsetroundjoin%
\definecolor{currentfill}{rgb}{0.000000,0.000000,0.000000}%
\pgfsetfillcolor{currentfill}%
\pgfsetlinewidth{0.803000pt}%
\definecolor{currentstroke}{rgb}{0.000000,0.000000,0.000000}%
\pgfsetstrokecolor{currentstroke}%
\pgfsetdash{}{0pt}%
\pgfsys@defobject{currentmarker}{\pgfqpoint{0.000000in}{-0.048611in}}{\pgfqpoint{0.000000in}{0.000000in}}{%
\pgfpathmoveto{\pgfqpoint{0.000000in}{0.000000in}}%
\pgfpathlineto{\pgfqpoint{0.000000in}{-0.048611in}}%
\pgfusepath{stroke,fill}%
}%
\begin{pgfscope}%
\pgfsys@transformshift{2.829091in}{0.528000in}%
\pgfsys@useobject{currentmarker}{}%
\end{pgfscope}%
\end{pgfscope}%
\begin{pgfscope}%
\definecolor{textcolor}{rgb}{0.000000,0.000000,0.000000}%
\pgfsetstrokecolor{textcolor}%
\pgfsetfillcolor{textcolor}%
\pgftext[x=2.829091in,y=0.430778in,,top]{\color{textcolor}\rmfamily\fontsize{10.000000}{12.000000}\selectfont \(\displaystyle {4}\)}%
\end{pgfscope}%
\begin{pgfscope}%
\pgfsetbuttcap%
\pgfsetroundjoin%
\definecolor{currentfill}{rgb}{0.000000,0.000000,0.000000}%
\pgfsetfillcolor{currentfill}%
\pgfsetlinewidth{0.803000pt}%
\definecolor{currentstroke}{rgb}{0.000000,0.000000,0.000000}%
\pgfsetstrokecolor{currentstroke}%
\pgfsetdash{}{0pt}%
\pgfsys@defobject{currentmarker}{\pgfqpoint{0.000000in}{-0.048611in}}{\pgfqpoint{0.000000in}{0.000000in}}{%
\pgfpathmoveto{\pgfqpoint{0.000000in}{0.000000in}}%
\pgfpathlineto{\pgfqpoint{0.000000in}{-0.048611in}}%
\pgfusepath{stroke,fill}%
}%
\begin{pgfscope}%
\pgfsys@transformshift{3.730909in}{0.528000in}%
\pgfsys@useobject{currentmarker}{}%
\end{pgfscope}%
\end{pgfscope}%
\begin{pgfscope}%
\definecolor{textcolor}{rgb}{0.000000,0.000000,0.000000}%
\pgfsetstrokecolor{textcolor}%
\pgfsetfillcolor{textcolor}%
\pgftext[x=3.730909in,y=0.430778in,,top]{\color{textcolor}\rmfamily\fontsize{10.000000}{12.000000}\selectfont \(\displaystyle {6}\)}%
\end{pgfscope}%
\begin{pgfscope}%
\pgfsetbuttcap%
\pgfsetroundjoin%
\definecolor{currentfill}{rgb}{0.000000,0.000000,0.000000}%
\pgfsetfillcolor{currentfill}%
\pgfsetlinewidth{0.803000pt}%
\definecolor{currentstroke}{rgb}{0.000000,0.000000,0.000000}%
\pgfsetstrokecolor{currentstroke}%
\pgfsetdash{}{0pt}%
\pgfsys@defobject{currentmarker}{\pgfqpoint{0.000000in}{-0.048611in}}{\pgfqpoint{0.000000in}{0.000000in}}{%
\pgfpathmoveto{\pgfqpoint{0.000000in}{0.000000in}}%
\pgfpathlineto{\pgfqpoint{0.000000in}{-0.048611in}}%
\pgfusepath{stroke,fill}%
}%
\begin{pgfscope}%
\pgfsys@transformshift{4.632727in}{0.528000in}%
\pgfsys@useobject{currentmarker}{}%
\end{pgfscope}%
\end{pgfscope}%
\begin{pgfscope}%
\definecolor{textcolor}{rgb}{0.000000,0.000000,0.000000}%
\pgfsetstrokecolor{textcolor}%
\pgfsetfillcolor{textcolor}%
\pgftext[x=4.632727in,y=0.430778in,,top]{\color{textcolor}\rmfamily\fontsize{10.000000}{12.000000}\selectfont \(\displaystyle {8}\)}%
\end{pgfscope}%
\begin{pgfscope}%
\pgfsetbuttcap%
\pgfsetroundjoin%
\definecolor{currentfill}{rgb}{0.000000,0.000000,0.000000}%
\pgfsetfillcolor{currentfill}%
\pgfsetlinewidth{0.803000pt}%
\definecolor{currentstroke}{rgb}{0.000000,0.000000,0.000000}%
\pgfsetstrokecolor{currentstroke}%
\pgfsetdash{}{0pt}%
\pgfsys@defobject{currentmarker}{\pgfqpoint{0.000000in}{-0.048611in}}{\pgfqpoint{0.000000in}{0.000000in}}{%
\pgfpathmoveto{\pgfqpoint{0.000000in}{0.000000in}}%
\pgfpathlineto{\pgfqpoint{0.000000in}{-0.048611in}}%
\pgfusepath{stroke,fill}%
}%
\begin{pgfscope}%
\pgfsys@transformshift{5.534545in}{0.528000in}%
\pgfsys@useobject{currentmarker}{}%
\end{pgfscope}%
\end{pgfscope}%
\begin{pgfscope}%
\definecolor{textcolor}{rgb}{0.000000,0.000000,0.000000}%
\pgfsetstrokecolor{textcolor}%
\pgfsetfillcolor{textcolor}%
\pgftext[x=5.534545in,y=0.430778in,,top]{\color{textcolor}\rmfamily\fontsize{10.000000}{12.000000}\selectfont \(\displaystyle {10}\)}%
\end{pgfscope}%
\begin{pgfscope}%
\definecolor{textcolor}{rgb}{0.000000,0.000000,0.000000}%
\pgfsetstrokecolor{textcolor}%
\pgfsetfillcolor{textcolor}%
\pgftext[x=3.280000in,y=0.251889in,,top]{\color{textcolor}\rmfamily\fontsize{10.000000}{12.000000}\selectfont \(\displaystyle t\) [s]}%
\end{pgfscope}%
\begin{pgfscope}%
\pgfsetbuttcap%
\pgfsetroundjoin%
\definecolor{currentfill}{rgb}{0.000000,0.000000,0.000000}%
\pgfsetfillcolor{currentfill}%
\pgfsetlinewidth{0.803000pt}%
\definecolor{currentstroke}{rgb}{0.000000,0.000000,0.000000}%
\pgfsetstrokecolor{currentstroke}%
\pgfsetdash{}{0pt}%
\pgfsys@defobject{currentmarker}{\pgfqpoint{-0.048611in}{0.000000in}}{\pgfqpoint{-0.000000in}{0.000000in}}{%
\pgfpathmoveto{\pgfqpoint{-0.000000in}{0.000000in}}%
\pgfpathlineto{\pgfqpoint{-0.048611in}{0.000000in}}%
\pgfusepath{stroke,fill}%
}%
\begin{pgfscope}%
\pgfsys@transformshift{0.800000in}{1.084767in}%
\pgfsys@useobject{currentmarker}{}%
\end{pgfscope}%
\end{pgfscope}%
\begin{pgfscope}%
\definecolor{textcolor}{rgb}{0.000000,0.000000,0.000000}%
\pgfsetstrokecolor{textcolor}%
\pgfsetfillcolor{textcolor}%
\pgftext[x=0.525308in, y=1.036573in, left, base]{\color{textcolor}\rmfamily\fontsize{10.000000}{12.000000}\selectfont \(\displaystyle {1.0}\)}%
\end{pgfscope}%
\begin{pgfscope}%
\pgfsetbuttcap%
\pgfsetroundjoin%
\definecolor{currentfill}{rgb}{0.000000,0.000000,0.000000}%
\pgfsetfillcolor{currentfill}%
\pgfsetlinewidth{0.803000pt}%
\definecolor{currentstroke}{rgb}{0.000000,0.000000,0.000000}%
\pgfsetstrokecolor{currentstroke}%
\pgfsetdash{}{0pt}%
\pgfsys@defobject{currentmarker}{\pgfqpoint{-0.048611in}{0.000000in}}{\pgfqpoint{-0.000000in}{0.000000in}}{%
\pgfpathmoveto{\pgfqpoint{-0.000000in}{0.000000in}}%
\pgfpathlineto{\pgfqpoint{-0.048611in}{0.000000in}}%
\pgfusepath{stroke,fill}%
}%
\begin{pgfscope}%
\pgfsys@transformshift{0.800000in}{1.679240in}%
\pgfsys@useobject{currentmarker}{}%
\end{pgfscope}%
\end{pgfscope}%
\begin{pgfscope}%
\definecolor{textcolor}{rgb}{0.000000,0.000000,0.000000}%
\pgfsetstrokecolor{textcolor}%
\pgfsetfillcolor{textcolor}%
\pgftext[x=0.525308in, y=1.631046in, left, base]{\color{textcolor}\rmfamily\fontsize{10.000000}{12.000000}\selectfont \(\displaystyle {1.2}\)}%
\end{pgfscope}%
\begin{pgfscope}%
\pgfsetbuttcap%
\pgfsetroundjoin%
\definecolor{currentfill}{rgb}{0.000000,0.000000,0.000000}%
\pgfsetfillcolor{currentfill}%
\pgfsetlinewidth{0.803000pt}%
\definecolor{currentstroke}{rgb}{0.000000,0.000000,0.000000}%
\pgfsetstrokecolor{currentstroke}%
\pgfsetdash{}{0pt}%
\pgfsys@defobject{currentmarker}{\pgfqpoint{-0.048611in}{0.000000in}}{\pgfqpoint{-0.000000in}{0.000000in}}{%
\pgfpathmoveto{\pgfqpoint{-0.000000in}{0.000000in}}%
\pgfpathlineto{\pgfqpoint{-0.048611in}{0.000000in}}%
\pgfusepath{stroke,fill}%
}%
\begin{pgfscope}%
\pgfsys@transformshift{0.800000in}{2.273713in}%
\pgfsys@useobject{currentmarker}{}%
\end{pgfscope}%
\end{pgfscope}%
\begin{pgfscope}%
\definecolor{textcolor}{rgb}{0.000000,0.000000,0.000000}%
\pgfsetstrokecolor{textcolor}%
\pgfsetfillcolor{textcolor}%
\pgftext[x=0.525308in, y=2.225519in, left, base]{\color{textcolor}\rmfamily\fontsize{10.000000}{12.000000}\selectfont \(\displaystyle {1.4}\)}%
\end{pgfscope}%
\begin{pgfscope}%
\pgfsetbuttcap%
\pgfsetroundjoin%
\definecolor{currentfill}{rgb}{0.000000,0.000000,0.000000}%
\pgfsetfillcolor{currentfill}%
\pgfsetlinewidth{0.803000pt}%
\definecolor{currentstroke}{rgb}{0.000000,0.000000,0.000000}%
\pgfsetstrokecolor{currentstroke}%
\pgfsetdash{}{0pt}%
\pgfsys@defobject{currentmarker}{\pgfqpoint{-0.048611in}{0.000000in}}{\pgfqpoint{-0.000000in}{0.000000in}}{%
\pgfpathmoveto{\pgfqpoint{-0.000000in}{0.000000in}}%
\pgfpathlineto{\pgfqpoint{-0.048611in}{0.000000in}}%
\pgfusepath{stroke,fill}%
}%
\begin{pgfscope}%
\pgfsys@transformshift{0.800000in}{2.868186in}%
\pgfsys@useobject{currentmarker}{}%
\end{pgfscope}%
\end{pgfscope}%
\begin{pgfscope}%
\definecolor{textcolor}{rgb}{0.000000,0.000000,0.000000}%
\pgfsetstrokecolor{textcolor}%
\pgfsetfillcolor{textcolor}%
\pgftext[x=0.525308in, y=2.819992in, left, base]{\color{textcolor}\rmfamily\fontsize{10.000000}{12.000000}\selectfont \(\displaystyle {1.6}\)}%
\end{pgfscope}%
\begin{pgfscope}%
\pgfsetbuttcap%
\pgfsetroundjoin%
\definecolor{currentfill}{rgb}{0.000000,0.000000,0.000000}%
\pgfsetfillcolor{currentfill}%
\pgfsetlinewidth{0.803000pt}%
\definecolor{currentstroke}{rgb}{0.000000,0.000000,0.000000}%
\pgfsetstrokecolor{currentstroke}%
\pgfsetdash{}{0pt}%
\pgfsys@defobject{currentmarker}{\pgfqpoint{-0.048611in}{0.000000in}}{\pgfqpoint{-0.000000in}{0.000000in}}{%
\pgfpathmoveto{\pgfqpoint{-0.000000in}{0.000000in}}%
\pgfpathlineto{\pgfqpoint{-0.048611in}{0.000000in}}%
\pgfusepath{stroke,fill}%
}%
\begin{pgfscope}%
\pgfsys@transformshift{0.800000in}{3.462659in}%
\pgfsys@useobject{currentmarker}{}%
\end{pgfscope}%
\end{pgfscope}%
\begin{pgfscope}%
\definecolor{textcolor}{rgb}{0.000000,0.000000,0.000000}%
\pgfsetstrokecolor{textcolor}%
\pgfsetfillcolor{textcolor}%
\pgftext[x=0.525308in, y=3.414465in, left, base]{\color{textcolor}\rmfamily\fontsize{10.000000}{12.000000}\selectfont \(\displaystyle {1.8}\)}%
\end{pgfscope}%
\begin{pgfscope}%
\pgfsetbuttcap%
\pgfsetroundjoin%
\definecolor{currentfill}{rgb}{0.000000,0.000000,0.000000}%
\pgfsetfillcolor{currentfill}%
\pgfsetlinewidth{0.803000pt}%
\definecolor{currentstroke}{rgb}{0.000000,0.000000,0.000000}%
\pgfsetstrokecolor{currentstroke}%
\pgfsetdash{}{0pt}%
\pgfsys@defobject{currentmarker}{\pgfqpoint{-0.048611in}{0.000000in}}{\pgfqpoint{-0.000000in}{0.000000in}}{%
\pgfpathmoveto{\pgfqpoint{-0.000000in}{0.000000in}}%
\pgfpathlineto{\pgfqpoint{-0.048611in}{0.000000in}}%
\pgfusepath{stroke,fill}%
}%
\begin{pgfscope}%
\pgfsys@transformshift{0.800000in}{4.057132in}%
\pgfsys@useobject{currentmarker}{}%
\end{pgfscope}%
\end{pgfscope}%
\begin{pgfscope}%
\definecolor{textcolor}{rgb}{0.000000,0.000000,0.000000}%
\pgfsetstrokecolor{textcolor}%
\pgfsetfillcolor{textcolor}%
\pgftext[x=0.525308in, y=4.008938in, left, base]{\color{textcolor}\rmfamily\fontsize{10.000000}{12.000000}\selectfont \(\displaystyle {2.0}\)}%
\end{pgfscope}%
\begin{pgfscope}%
\definecolor{textcolor}{rgb}{0.000000,0.000000,0.000000}%
\pgfsetstrokecolor{textcolor}%
\pgfsetfillcolor{textcolor}%
\pgftext[x=0.469752in,y=2.376000in,,bottom,rotate=90.000000]{\color{textcolor}\rmfamily\fontsize{10.000000}{12.000000}\selectfont \(\displaystyle V_c(t)\) [V]}%
\end{pgfscope}%
\begin{pgfscope}%
\pgfpathrectangle{\pgfqpoint{0.800000in}{0.528000in}}{\pgfqpoint{4.960000in}{3.696000in}}%
\pgfusepath{clip}%
\pgfsetrectcap%
\pgfsetroundjoin%
\pgfsetlinewidth{1.505625pt}%
\definecolor{currentstroke}{rgb}{0.121569,0.466667,0.705882}%
\pgfsetstrokecolor{currentstroke}%
\pgfsetdash{}{0pt}%
\pgfpathmoveto{\pgfqpoint{1.025455in}{1.084767in}}%
\pgfpathlineto{\pgfqpoint{1.061600in}{1.313733in}}%
\pgfpathlineto{\pgfqpoint{1.097745in}{1.525062in}}%
\pgfpathlineto{\pgfqpoint{1.133890in}{1.720111in}}%
\pgfpathlineto{\pgfqpoint{1.170035in}{1.900136in}}%
\pgfpathlineto{\pgfqpoint{1.206180in}{2.066293in}}%
\pgfpathlineto{\pgfqpoint{1.242325in}{2.219650in}}%
\pgfpathlineto{\pgfqpoint{1.278470in}{2.361195in}}%
\pgfpathlineto{\pgfqpoint{1.314615in}{2.491836in}}%
\pgfpathlineto{\pgfqpoint{1.350760in}{2.612413in}}%
\pgfpathlineto{\pgfqpoint{1.386905in}{2.723702in}}%
\pgfpathlineto{\pgfqpoint{1.423050in}{2.826418in}}%
\pgfpathlineto{\pgfqpoint{1.459195in}{2.921222in}}%
\pgfpathlineto{\pgfqpoint{1.495340in}{3.008723in}}%
\pgfpathlineto{\pgfqpoint{1.531485in}{3.089484in}}%
\pgfpathlineto{\pgfqpoint{1.567630in}{3.164023in}}%
\pgfpathlineto{\pgfqpoint{1.603775in}{3.232821in}}%
\pgfpathlineto{\pgfqpoint{1.639920in}{3.296319in}}%
\pgfpathlineto{\pgfqpoint{1.676065in}{3.354926in}}%
\pgfpathlineto{\pgfqpoint{1.712210in}{3.409018in}}%
\pgfpathlineto{\pgfqpoint{1.748355in}{3.458943in}}%
\pgfpathlineto{\pgfqpoint{1.784500in}{3.505023in}}%
\pgfpathlineto{\pgfqpoint{1.820645in}{3.547552in}}%
\pgfpathlineto{\pgfqpoint{1.856790in}{3.586806in}}%
\pgfpathlineto{\pgfqpoint{1.892935in}{3.623036in}}%
\pgfpathlineto{\pgfqpoint{1.920044in}{3.648365in}}%
\pgfpathlineto{\pgfqpoint{1.929080in}{0.696000in}}%
\pgfpathlineto{\pgfqpoint{1.965225in}{0.954914in}}%
\pgfpathlineto{\pgfqpoint{2.001370in}{1.193882in}}%
\pgfpathlineto{\pgfqpoint{2.037515in}{1.414443in}}%
\pgfpathlineto{\pgfqpoint{2.073660in}{1.618014in}}%
\pgfpathlineto{\pgfqpoint{2.109805in}{1.805903in}}%
\pgfpathlineto{\pgfqpoint{2.145950in}{1.979319in}}%
\pgfpathlineto{\pgfqpoint{2.182095in}{2.139376in}}%
\pgfpathlineto{\pgfqpoint{2.218240in}{2.287104in}}%
\pgfpathlineto{\pgfqpoint{2.254385in}{2.423452in}}%
\pgfpathlineto{\pgfqpoint{2.290530in}{2.549297in}}%
\pgfpathlineto{\pgfqpoint{2.326675in}{2.665448in}}%
\pgfpathlineto{\pgfqpoint{2.362820in}{2.772652in}}%
\pgfpathlineto{\pgfqpoint{2.398965in}{2.871598in}}%
\pgfpathlineto{\pgfqpoint{2.435110in}{2.962921in}}%
\pgfpathlineto{\pgfqpoint{2.471255in}{3.047210in}}%
\pgfpathlineto{\pgfqpoint{2.507400in}{3.125006in}}%
\pgfpathlineto{\pgfqpoint{2.543545in}{3.196809in}}%
\pgfpathlineto{\pgfqpoint{2.579690in}{3.263081in}}%
\pgfpathlineto{\pgfqpoint{2.615835in}{3.324248in}}%
\pgfpathlineto{\pgfqpoint{2.651980in}{3.380704in}}%
\pgfpathlineto{\pgfqpoint{2.688125in}{3.432810in}}%
\pgfpathlineto{\pgfqpoint{2.724270in}{3.480903in}}%
\pgfpathlineto{\pgfqpoint{2.760415in}{3.525290in}}%
\pgfpathlineto{\pgfqpoint{2.796560in}{3.566259in}}%
\pgfpathlineto{\pgfqpoint{2.832705in}{3.604072in}}%
\pgfpathlineto{\pgfqpoint{2.868850in}{3.638972in}}%
\pgfpathlineto{\pgfqpoint{2.904995in}{3.671183in}}%
\pgfpathlineto{\pgfqpoint{2.941140in}{3.700914in}}%
\pgfpathlineto{\pgfqpoint{2.986322in}{3.734877in}}%
\pgfpathlineto{\pgfqpoint{3.031503in}{3.765602in}}%
\pgfpathlineto{\pgfqpoint{3.076684in}{3.793398in}}%
\pgfpathlineto{\pgfqpoint{3.121866in}{3.818543in}}%
\pgfpathlineto{\pgfqpoint{3.167047in}{3.841291in}}%
\pgfpathlineto{\pgfqpoint{3.212228in}{3.861870in}}%
\pgfpathlineto{\pgfqpoint{3.266446in}{3.883992in}}%
\pgfpathlineto{\pgfqpoint{3.320663in}{3.903607in}}%
\pgfpathlineto{\pgfqpoint{3.374881in}{3.921001in}}%
\pgfpathlineto{\pgfqpoint{3.438134in}{3.938818in}}%
\pgfpathlineto{\pgfqpoint{3.501388in}{3.954304in}}%
\pgfpathlineto{\pgfqpoint{3.573678in}{3.969536in}}%
\pgfpathlineto{\pgfqpoint{3.645968in}{3.982511in}}%
\pgfpathlineto{\pgfqpoint{3.727295in}{3.994826in}}%
\pgfpathlineto{\pgfqpoint{3.817657in}{4.006141in}}%
\pgfpathlineto{\pgfqpoint{3.917056in}{4.016229in}}%
\pgfpathlineto{\pgfqpoint{4.025491in}{4.024972in}}%
\pgfpathlineto{\pgfqpoint{4.151999in}{4.032839in}}%
\pgfpathlineto{\pgfqpoint{4.296579in}{4.039503in}}%
\pgfpathlineto{\pgfqpoint{4.468267in}{4.045086in}}%
\pgfpathlineto{\pgfqpoint{4.676101in}{4.049534in}}%
\pgfpathlineto{\pgfqpoint{4.947189in}{4.052967in}}%
\pgfpathlineto{\pgfqpoint{5.326712in}{4.055337in}}%
\pgfpathlineto{\pgfqpoint{5.534545in}{4.056000in}}%
\pgfpathlineto{\pgfqpoint{5.534545in}{4.056000in}}%
\pgfusepath{stroke}%
\end{pgfscope}%
\begin{pgfscope}%
\pgfsetrectcap%
\pgfsetmiterjoin%
\pgfsetlinewidth{0.803000pt}%
\definecolor{currentstroke}{rgb}{0.000000,0.000000,0.000000}%
\pgfsetstrokecolor{currentstroke}%
\pgfsetdash{}{0pt}%
\pgfpathmoveto{\pgfqpoint{0.800000in}{0.528000in}}%
\pgfpathlineto{\pgfqpoint{0.800000in}{4.224000in}}%
\pgfusepath{stroke}%
\end{pgfscope}%
\begin{pgfscope}%
\pgfsetrectcap%
\pgfsetmiterjoin%
\pgfsetlinewidth{0.803000pt}%
\definecolor{currentstroke}{rgb}{0.000000,0.000000,0.000000}%
\pgfsetstrokecolor{currentstroke}%
\pgfsetdash{}{0pt}%
\pgfpathmoveto{\pgfqpoint{5.760000in}{0.528000in}}%
\pgfpathlineto{\pgfqpoint{5.760000in}{4.224000in}}%
\pgfusepath{stroke}%
\end{pgfscope}%
\begin{pgfscope}%
\pgfsetrectcap%
\pgfsetmiterjoin%
\pgfsetlinewidth{0.803000pt}%
\definecolor{currentstroke}{rgb}{0.000000,0.000000,0.000000}%
\pgfsetstrokecolor{currentstroke}%
\pgfsetdash{}{0pt}%
\pgfpathmoveto{\pgfqpoint{0.800000in}{0.528000in}}%
\pgfpathlineto{\pgfqpoint{5.760000in}{0.528000in}}%
\pgfusepath{stroke}%
\end{pgfscope}%
\begin{pgfscope}%
\pgfsetrectcap%
\pgfsetmiterjoin%
\pgfsetlinewidth{0.803000pt}%
\definecolor{currentstroke}{rgb}{0.000000,0.000000,0.000000}%
\pgfsetstrokecolor{currentstroke}%
\pgfsetdash{}{0pt}%
\pgfpathmoveto{\pgfqpoint{0.800000in}{4.224000in}}%
\pgfpathlineto{\pgfqpoint{5.760000in}{4.224000in}}%
\pgfusepath{stroke}%
\end{pgfscope}%
\end{pgfpicture}%
\makeatother%
\endgroup%
}
\end{center}

\question{}

\begin{align}
    V_{cm} &= \frac{V_1 + V_2}{2} \\
    V_d &= V_2 - V_1 \\
    V_{out} &= A_{cm} V_{cm} + A_d V_d
\end{align}
We can solve for \(V_1\) and \(V_2\) in terms of \(V_{cm}\) and \(V_d\) as
\begin{align}
    V_1 &= V_{cm} - \frac{V_d}{2} \\
    V_2 &= V_{cm} + \frac{V_d}{2}
\end{align}
Then, we can find \(V_{out}\) as
\begin{align}
    V_{out} &= -\cancelto{10}{\frac{R_3}{R_1}} V_1 + \left(1 + \cancelto{10}{\frac{R_3}{R_1}}\right) \left(\frac{R_4}{R_2 + R_4}\right) V_2 \\
            &= -10 V_1 + 11 \cancelto{\frac{10}{11}}{\left(1 + \frac{R_2}{R_4}\right)^{-1}} V_2 = -10 V_1 + 10 V_2 \\
            &= -10 \left(\cancel{V_{cm}} - \frac{V_d}{2}\right) + 10 \left(\cancel{V_{cm}} + \frac{V_d}{2}\right) = 10 V_d
\end{align}
So \(A_{cm} = 0\) and \(A_d = 10\).

\end{document}
